\newglossaryentry{lpfc}
{
  name=lPFC,
  description={lateral Prefrontal Cortex.},
  first={lateral Prefrontal Cortex (lPFC)}
}

\newglossaryentry{dlpfc}
{
  name=dlPFC,
  description={dorsolateral Prefrontal Cortex.},
  first={dorsolateral Prefrontal Cortex (dlPFC)}
}

\newglossaryentry{vlpfc}
{
  name=vlPFC,
  description={ventrolateral Prefrontal Cortex.},
  first={ventrolateral Prefrontal Cortex (vlPFC)}
}

\newglossaryentry{mds}
{
  name=MDS,
  description={Multi Dimensional Scaling. A mathematical technique to visualize data in an arbitrary number of dimensions.},
  first={Multi Dimensional Scaling (vlPFC)}
}


\newglossaryentry{pfc}
{
  name=PFC,
  description={Prefrontal Cortex.},
  first={Prefrontal Cortex (PFC)}
}

\newglossaryentry{wcs}
{
  name=WCS,
  description={Wisconsin Card Sorting.},
  first={Wisconsin Card Sorting (WCS)}
}

\newglossaryentry{bg}
{
  name=BG,
  description={Basal Ganglia.},
  first={Basal Ganglia (BG)}
}

\newglossaryentry{acc}
{
  name=ACC,
  description={Anterior Cingulate Cortex.},
  first={Anterior Cingulate Cortex (ACC)}
}

\newglossaryentry{sma}
{
  name=SMA,
  description={Supplementary Motor Area.},
  first={Supplementary Motor Area (SMA)}
}

\newglossaryentry{eec}
{
  name=EEC,
  description={Embodied and Embedded Cognition, also known as Embodied and Situated Cognition. },
  first={Embodied and Embedded Cognition (EEC)}
}

\newglossaryentry{spa}
{
  name=SPA,
  description={Sense - Plan - Act. One of the design paradigms in artificial intelligence and robotics, especially in GOFAI. },
  first={Sense - Plan - Act (SPA)}
}

\newglossaryentry{gofai}
{
  name=GOFAI,
  description={Good Old Fashioned Artificial Intelligence. },
  first={Good Old Fashioned Artificial Intelligence (GOFAI)}
}

\newglossaryentry{esn}
{
  name=ESN,
  description={Echo State Network. A type of Reservoir Computing that employs coarsely integrated Continuous Time Recurrent Neural Networks as its reservoir.},
  first={Echo State Network (ESN)}
}

\newglossaryentry{rc}
{
  name=RC,
  description={Reservoir Computing. Solving problems by feeding input signals into a (non-linear) reservoir and probe its response. Echo State Networks and Liquid State Machines are two subtypes.},
  first={Reservoir Computing (RC)}
}

\newglossaryentry{ea}
{
  name=EA,
  description={Evolutionary Algorithm},
  first={Evolutionary Algorithm (EA)}
}

\newglossaryentry{lsm}
{
  name=LSM,
  description={Liquid State Machine. A type of Reservoir Computing that employs spiking neural networks as its reservoir.},
  first={Liquid State Machine (LSM)}
}

\newglossaryentry{rtrl}
{
  name=RTRL,
  description={Real-Time Recurrent Learning. This is one of the supervised learning algorithms for recurrent neural networks.},
  first={Real-Time Recurrent Learning (RTRL)}
}

\newglossaryentry{bptt}
{
  name=BPTT,
  description={Backpropagation Trough Time. This is one of the supervised learning algorithms for recurrent neural networks.},
  first={Backpropagation Trough Time (BPTT)}
}

\newglossaryentry{ctrnn}
{
  name=CTRNN,
  description={Continuous Time Recurrent Neural Network.},
  first={Continuous Time Recurrent Neural Network (CTRNN)}
}

\newglossaryentry{designmatrix}
{
  name={Design Matrix},
  description={The matrix that contains all the information that will be presented to a participant during an experiment. One dimension corresponds to time and every element in this dimension can correspond to a trial (or, alternatively, another arbitrary step in time). The other dimension consists of the different signals at that point in time, for example the input signals.}  
}

\newglossaryentry{ba}
{
  name={BA},
  description={Brodmann Area. The de facto standard in brain topology, based on cytoarchitectonic differences. },
  first={Brodmann Area (BA)}
}

\newglossaryentry{dorsal}
{
  name={Dorsal},
  description={Towards the top, superior.}, 
}

\newglossaryentry{ventral}
{
  name={Ventral},
  description={Towards the belly, inferior.},
}

\newglossaryentry{rostral}
{
  name={Rostral},
  description={Towards the front, frontal, anterior.},
}

\newglossaryentry{anterior}
{
  name={Anterior},
  description={Towards the front, frontal, rostral.},
}

\newglossaryentry{lateral}
{
  name={Lateral},
  description={Towards the side. },
}

\newglossaryentry{medial}
{
  name={Medial},
  description={Towards the midline.},
}

\newglossaryentry{caudal}
{
  name={Caudal},
  description={Towards the back, posterior.},
}

\newglossaryentry{posterior}
{
  name={Posterior},
  description={Towards the back, caudal.},
}

\renewcommand*{\glspostdescription}{} %no dot at the end
\renewcommand{\glossarysection}[2][]{} %no new section/chapter
\renewcommand{\glsgroupskip}{} %vertical spacing
\setlength{\glsdescwidth}{0.83\linewidth} %make the glossary a bit wider
\makeglossaries
